\documentclass[a4paper]{jpconf}
\usepackage{graphicx}
\begin{document}
\title{Toward real-time data query systems in HEP}

\author{Jim Pivarski}

\address{147 N.\ Ridgeland Ave.\ Unit 3S, Oak Park, IL 60302, US}

\ead{pivarski@fnal.gov}

\begin{abstract}
Exploratory data analysis tools must respond quickly to a user's questions, so that the answer to one question (e.g.\ a visualized distribution) can influence the next. In some SQL-based query systems used in industry, even very large (petabyte) datasets can be summarized on a human timescale (seconds), emplying techniques like columnar data representation, caching, indexing, and code generation/JIT-compilation. This article describes progress toward realizing such a system for HEP, focusing on the intermediate problems of optimizing data access and calculations for ``query sized'' payloads, such as a single histogram or group of histograms, rather than large reconstruction or data-skimming jobs. These techniques include direct extraction of ROOT TBranches into Numpy arrays and compilation of Python analysis functions (rather than SQL) to be executed very quickly. We will also discuss the problem of caching and actively delivering jobs to worker nodes with input data preloaded. All of these pieces of the larger solution are available as standalone GitHub repositories, and could be used in current analyses.
\end{abstract}

\section{Introduction}





\end{document}
